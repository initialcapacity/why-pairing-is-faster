\documentclass[letterpaper]{article}
\usepackage[pass]{geometry}
\usepackage{amsthm}

\newtheorem{theorem}{Theorem}[section]
\newtheorem{lemma}[theorem]{Lemma}
\newtheorem{corollary}[theorem]{Corollary}

\theoremstyle{definition}
\newtheorem{definition}[theorem]{Definition}

\title{Why Pairing is Faster}
\author{Michael Barinek \& Tyson Gern}
\date{\today}

\begin{document}
    \maketitle

    \begin{abstract}
        We have long known that pairing enhances knowledge sharing, ramping, and training.
        These benefits outweighed the previously-assumed hit to overall velocity due to having two people do the work of
        one.
        Recently, we have learned that this impact is not as great as once thought, and that pairing can even improve
        velocity in the short term.
        We complete more features more quickly by pairing because the code is reviewed in real time, rather than
        asynchronously.
    \end{abstract}

    \section{Introduction}\label{sec:introduction}

    Learning a new language or domain on your own versus working with a seasoned developer;
    a developer that can share lessons learned or clarify a slightly complex domain nomenclature.
    The hypothesis being that a better understanding should naturally result in higher productivity or developing
    software faster.

    However, the question quickly becomes: What happens when the team understands both product domain and technology?
    Should the team still pair?

    The simple answer might be yes, because both domain and technology are constantly changing.
    There is always more to learn, and now we have proof that pairing ships product faster.

    \section{Workflows}\label{sec:workflows}

    We begin with a series of fundamental definitions.

    \begin{definition}
        A \textit{user story} is an independently-deployable unit of work that delivers user value.
    \end{definition}

    \begin{definition}
        A \textit{well-functioning team} splits their work into the smallest user stories as possible in order to stay
        flexible and deliver incremental value.
        The code for each user story should be able to be completed within four to eight hours.
    \end{definition}

    \begin{definition}
        The \textit{point estimate} of a user story represents the well-functioning team's judgment of the story's
        complexity.
    \end{definition}

    While point estimates are not tied directly to time, the number of points that a well-functioning team can complete
    in a week is relatively constant.
    For the sake of simplicity we'll refer to $n$-hour stories.

    \begin{definition}
        A user story is \textit{finished} once the code is merged into the mainline branch.
    \end{definition}

    \begin{definition}
        The \textit{pairing workflow} consists of developers working in pairs, conducting live code review, and
        committing to the mainline branch once the code is complete.
    \end{definition}

    \begin{lemma}\label{lemma:pair}
        Two developers working in the pairing workflow will finish an $n$-hour story in $n$ hours.
    \end{lemma}
    \begin{proof}
        Each $n$-hour story takes $n$ hours to code.
        Since code review is completed live while pairing, the code is merged into the mainline branch in $n$ hours.
    \end{proof}

    \begin{definition}
        In the \textit{pull request workflow} each developer works individually and submits the code for each user story
        asynchronous review by others through pull requests.
        Another developer reviews thoroughly reviews the pull request and suggests changes.
        The author reviews these changes, updates the pull request, and merges the code into the mainline branch.
    \end{definition}

    \begin{lemma}\label{lemma:solo}
        A developer working in the pull request workflow will complete an $n$-hour story in $2n$ hours.
    \end{lemma}
    \begin{proof}
        In order to complete a thorough code review, the reviewer spend at least $\frac{n}{4}$ hours reviewing the pull
        request, and the author must the same amount of time reviewing the changes and applying updates.
        To allow for the asynchronous nature of the pull request workflow, we must allow for at least $\frac{n}{2}$
        hours in waiting and communication time.
    \end{proof}

    \section{Parallelization}\label{sec:parallelization}

    \begin{definition}
        A \textit{backlog} is an ordered collection of user stories.
    \end{definition}

    \begin{definition}
        A user story $\alpha_1$ \textit{depends on} a user story $\alpha_2$ if developers must wait to start work on
        $\alpha_1$ until $\alpha_2$ is finished.
    \end{definition}

    \begin{definition}
        A backlog is \textit{parallelizable} into $n$ workstreams (at a given point in time) if there are $n$ stories
        able to be started, reviewed, or revised.
    \end{definition}

    The power of pairing is the ability for a team to work quickly on a backlog, even when parallelization is limited.

    \begin{definition}
        A team's \textit{velocity} is the speed at which it delivers user value.
        We measure velocity as the rate at which the teams finishes stories, where each story is weighted by its
        estimate.
        If, in a given day, a team completes stories $\{\alpha_1, \alpha_2,\dots,\alpha_k\}$ with estimates
        $\{n_1, n_2,\dots,n_k\}$ then the team's daily velocity is $\sum_1^k n_k$.
    \end{definition}

    \begin{lemma}\label{lemma:parallel}
        A team of $2n$ developers with a backlog with more than $2n$ workstreams has an average daily velocity of $4n$,
        regardless of working model (assuming an 8-hour day).
    \end{lemma}
    \begin{proof}
        By Lemma~\ref{lemma:pair} each pair will complete $8n$ worth of stories in an 8-hour day and by
        Lemma~\ref{lemma:solo} each developer will complete $4n$ worth of stories in an 8-hour day, giving each team an
        average daily velocity of $4n$.
    \end{proof}

    \begin{theorem}
        A well-functioning team of developers working as pairs is as fast or faster than a team of developers using pull
        requests.
    \end{theorem}
    \begin{proof}
        Suppose a team of $2n$ developers has a backlog that is parallelizable into $m > 0$ workstreams.
        We will examine three different cases for the size $m$, and show that the team's velocity working in pairs is at
        least as fast as the team's velocity using pull requests.

        Suppose that $m\geq 2n$.
        Then by Lemma~\ref{lemma:parallel} the team achieves an equal velocity regardless of working model.

        Next, suppose that $n\leq m < 2n$.
        In a team working in a pairing model, each of the $n$ pairs are able to work unrestricted, so the team has a
        daily velocity of $4n$.
        In a team working with pull requests, only $m$ developers are busy at any point in time, so the team has a daily
        velocity of $2m$.
        Using the original assumption that $m < 2n$ we see that $2m < 4n$, so the team working in the pairing model is
        faster.

        Finally, suppose that $0 < m < n$
        In a team working in a pairing model, $m$ pairs are able to work unrestricted, thus the team has a daily
        velocity of $4m$.
        In a team working with pull requests, only $m$ developers are busy at any point in time, so the team has a daily
        velocity of $2m$.
        As before, the team working in a pairing model is faster.
    \end{proof}
\end{document}
